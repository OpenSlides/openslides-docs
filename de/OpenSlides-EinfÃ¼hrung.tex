%% LyX 2.2.0 created this file.  For more info, see http://www.lyx.org/.
%% Do not edit unless you really know what you are doing.
\documentclass[12pt,a4paper,ngerman,bibliography=totoc,index=totoc,BCOR7.5mm,titlepage,captions=tableheading,dvipsnames,table]{scrbook}
\usepackage{lmodern}
\renewcommand{\sfdefault}{lmss}
\renewcommand{\ttdefault}{lmtt}
\usepackage[T1]{fontenc}
\usepackage[latin9]{inputenc}
\setcounter{secnumdepth}{3}
\setlength{\parskip}{\medskipamount}
\setlength{\parindent}{0pt}
\usepackage{color}
\usepackage{babel}
\usepackage{amsmath}
\usepackage{amssymb}
\usepackage[unicode=true,
 bookmarks=true,bookmarksnumbered=true,bookmarksopen=true,bookmarksopenlevel=1,
 breaklinks=true,pdfborder={0 0 1},backref=false,colorlinks=true]
 {hyperref}
\hypersetup{pdftitle={OpenSlides Handbuch},
 pdfauthor={OpenSlides Team},
 pdfsubject={OpenSlides Handbuch},
 pdfkeywords={OpenSlides, Handbuch},
 linkcolor=black, citecolor=black, urlcolor=blue, filecolor=blue, pdfpagelayout=OneColumn, pdfnewwindow=true, pdfstartview=XYZ, plainpages=false}

\makeatletter

%%%%%%%%%%%%%%%%%%%%%%%%%%%%%% LyX specific LaTeX commands.
\pdfpageheight\paperheight
\pdfpagewidth\paperwidth

\newcommand{\noun}[1]{\textsc{#1}}

\@ifundefined{date}{}{\date{}}
%%%%%%%%%%%%%%%%%%%%%%%%%%%%%% User specified LaTeX commands.
% that links to image floats jumps
% to the beginning of the float and 
% not to its caption
\usepackage[figure]{hypcap}

% the pages of the TOC is numbered roman
% and a PDF-bookmark for the TOC is added
\let\myTOC\tableofcontents
\renewcommand\tableofcontents{%
  \frontmatter
  \pdfbookmark[1]{\contentsname}{}
  \myTOC
  \mainmatter }

% Linkfl�che f�r Querverweise vergr��ern und automatisch benennen,
\AtBeginDocument{\renewcommand{\ref}[1]{\mbox{\autoref{#1}}}}
\@ifpackageloaded{babel}{
 \addto\extrasngerman{%
  \renewcommand*{\equationautorefname}[1]{}%
  \renewcommand{\sectionautorefname}{Kap.\negthinspace}%
  \renewcommand{\subsectionautorefname}{Kap.\negthinspace}%
  \renewcommand{\subsubsectionautorefname}{Kap.\negthinspace}%
 }
}{}

% provides caption formatting
\usepackage[labelfont={bf,sf}]{caption}[2004/07/16]

% enables calculation of values,
\usepackage{calc}

% increase the bottom float placement fraction
\renewcommand{\bottomfraction}{0.5}

% avoids that floats are placed before their
% corresponding section starts
\let\mySection\section\renewcommand{\section}{\suppressfloats[t]\mySection}

% used to have extra space in table cells
\@ifundefined{extrarowheight}
 {\usepackage{array}}{}
\setlength{\extrarowheight}{2pt}

\makeatother

\begin{document}

\chapter{Einf�hrung}

\noun{OpenSlides} ist ein freies, webbasiertes Pr�sentations- und
Versammlungssystem zur Darstellung und Steuerung von Tagesordnung,
Antr�gen und Wahlen einer Veranstaltung, insbesondere bei Einsatz
eines oder mehrerer Projektoren.

\noun{OpenSlides} wurde vor allem f�r Mitgliederversammlungen, Delegiertenversammlungen,
Hauptversammlungen oder Parteitage entwickelt. Mit \noun{OpenSlides}
k�nnen alle Inhalte der Veranstaltung zeitaktuell anhand des Verlaufs
der Veranstaltung angepasst und an die Leinwand projiziert werden.
Dies betrifft vor allem die Tagesordnung und aktuelle Informationen
f�r die Anwesenden. Zus�tzlich gibt es Funktionen f�r die Teilnehmerverwaltung,
Antr�ge, Abstimmungen und Wahlen.

\noun{OpenSlides} kann in drei Modi betrieben werden: 
\begin{description}
\item [{Pr�sentationsmodus:}] Im Pr�sentationsmodus steuern ein oder mehrere
Veranstalter (zum Beispiel Vorstandsmitglieder eines Vereins oder
einer Partei) das Programm allein. Sie bearbeiten diese Inhalte interaktiv
und bestimmen, was auf dem Projektor gezeigt wird. Alle Anwesenden
k�nnen die Veranstaltung auf der Leinwand verfolgen.
\item [{Pr�sentationsmodus~Single:}] Der Unterschied zum normalen Pr�sentationsmodus
ist, dass nur eine einzige Person das gesamte System bedient \emph{und}
der Projektor auch an ihrem Computer angeschlossen ist.
\item [{Teilnehmermodus:}] Im Teilnehmermodus k�nnen sich zus�tzlich zur
Versammlungsleitung die anwesenden Teilnehmer mit dem eigenen Laptop
oder Smartphone �ber Netzwerk (zum Beispiel WLAN) anmelden und eigene
Inhalte bearbeiten, zum Beispiel einen Antrag stellen oder einen Kandidaten
zur Wahl vorschlagen. Eine umfangreiche Rechteverwaltung garantiert
die notwendige Sicherheit des Systems. Die Versammlungsleitung hat
in der Regel weitgehende, die Versammlungsteilnehmer nur eingeschr�nkte
Zugriffsrechte.
\end{description}
Au�er beim Pr�sentationsmodus Single ben�tigt man ein Computernetzwerk,
zum Beispiel mit mehreren LAN-Anschl�ssen oder einem WLAN im Raum.
Ein Computer fungiert hier als Server f�r \noun{OpenSlides}. Alle
anderen Computer m�ssen auf dessen IP-Adresse zugreifen k�nnen. Der
Computer, an dem der Projektor angeschlossen ist, kann auch ein eigenst�ndiger
Computer sein, der �ber das Netzwerk auf den Server zugreift.

\noun{OpenSlides} ist webbasiert, das hei�t, es muss bei den Benutzern
und Teilnehmern keine neue Software installiert werden. Das Programm
wird ausschlie�lich �ber den eigenen Browser bedient. Alle g�ngigen
aktuellen Browser werden unterst�tzt. Lediglich die Veranstaltungsleitung
muss auf einem einzigen Computer, dem Server, \noun{OpenSlides} installieren
und starten.
\end{document}
