%% LyX 2.2.0 created this file.  For more info, see http://www.lyx.org/.
%% Do not edit unless you really know what you are doing.
\documentclass[12pt,a4paper,ngerman,bibliography=totoc,index=totoc,BCOR7.5mm,titlepage,captions=tableheading,dvipsnames,table]{scrbook}
\usepackage{lmodern}
\renewcommand{\sfdefault}{lmss}
\renewcommand{\ttdefault}{lmtt}
\usepackage[T1]{fontenc}
\usepackage[latin9]{inputenc}
\setcounter{secnumdepth}{3}
\setlength{\parskip}{\medskipamount}
\setlength{\parindent}{0pt}
\usepackage{color}
\usepackage{babel}
\usepackage{amsmath}
\usepackage{amssymb}
\usepackage[unicode=true,
 bookmarks=true,bookmarksnumbered=true,bookmarksopen=true,bookmarksopenlevel=1,
 breaklinks=true,pdfborder={0 0 1},backref=false,colorlinks=true]
 {hyperref}
\hypersetup{pdftitle={OpenSlides Handbuch},
 pdfauthor={OpenSlides Team},
 pdfsubject={OpenSlides Handbuch},
 pdfkeywords={OpenSlides, Handbuch},
 linkcolor=black, citecolor=black, urlcolor=blue, filecolor=blue, pdfpagelayout=OneColumn, pdfnewwindow=true, pdfstartview=XYZ, plainpages=false}

\makeatletter

%%%%%%%%%%%%%%%%%%%%%%%%%%%%%% LyX specific LaTeX commands.
\pdfpageheight\paperheight
\pdfpagewidth\paperwidth

\newcommand{\noun}[1]{\textsc{#1}}

\@ifundefined{date}{}{\date{}}
%%%%%%%%%%%%%%%%%%%%%%%%%%%%%% User specified LaTeX commands.
% that links to image floats jumps
% to the beginning of the float and 
% not to its caption
\usepackage[figure]{hypcap}

% the pages of the TOC is numbered roman
% and a PDF-bookmark for the TOC is added
\let\myTOC\tableofcontents
\renewcommand\tableofcontents{%
  \frontmatter
  \pdfbookmark[1]{\contentsname}{}
  \myTOC
  \mainmatter }

% Linkfl�che f�r Querverweise vergr��ern und automatisch benennen,
\AtBeginDocument{\renewcommand{\ref}[1]{\mbox{\autoref{#1}}}}
\@ifpackageloaded{babel}{
 \addto\extrasngerman{%
  \renewcommand*{\equationautorefname}[1]{}%
  \renewcommand{\sectionautorefname}{Kap.\negthinspace}%
  \renewcommand{\subsectionautorefname}{Kap.\negthinspace}%
  \renewcommand{\subsubsectionautorefname}{Kap.\negthinspace}%
 }
}{}

% provides caption formatting
\usepackage[labelfont={bf,sf}]{caption}[2004/07/16]

% enables calculation of values,
\usepackage{calc}

% increase the bottom float placement fraction
\renewcommand{\bottomfraction}{0.5}

% avoids that floats are placed before their
% corresponding section starts
\let\mySection\section\renewcommand{\section}{\suppressfloats[t]\mySection}

% used to have extra space in table cells
\@ifundefined{extrarowheight}
 {\usepackage{array}}{}
\setlength{\extrarowheight}{2pt}

\makeatother

\begin{document}

\chapter{Weitere Anpassungen von \protect\noun{OpenSlides}\label{chap:Weitere-Anpassungen-von}}

\section{Eigenes Logo einf�gen}

Um das Logo zu �ndern, muss die Datei \textbf{logo-projector.png}
im Unterordner\\
\textbf{\textasciitilde{}\textbackslash{}Lib\textbackslash{}site-packages\textbackslash{}openslides\textbackslash{}core\textbackslash{}static\textbackslash{}img}\\
des Verzeichnisses, in dem \noun{OpenSlides} installiert ist, durch
eine Datei gleichen Namens ersetzt werden.

\section{Plugins}

\noun{OpenSlides} kann mit Hilfe von Plugins erweitert werden. Plugins
dienen dazu bestimmte Funktionen hinzuzuf�gen, die standardm��ig nicht
verf�gbar sind. Ein Beispiel ist das in \ref{subsec:Das-Plugin-CSV}
beschriebene Plugin \emph{CSV Export Plugin for OpenSlides}.

Plugins werden installiert, indem man den Dateiordner des Plugins
komplett in den Unterordner\\
\textbf{\textasciitilde{}\textbackslash{}openslides\textbackslash{}plugins}\\
des Verzeichnisses, in dem \noun{OpenSlides} installiert ist, kopiert.
Falls \noun{OpenSlides} l�uft, muss es geschlossen und der Server
neu gestartet werden, wie es in \ref{sec:Start-des-Servers} beschrieben
ist. In der Men�leiste von \noun{OpenSlides} erscheint danach ein
neues Icon des entsprechenden Plugins.

Wie man Plugins f�r \noun{OpenSlides} programmiert, ist in ?? beschrieben.

\section{Eigene Datenbank anbinden}

Dieses Handbuch ist noch nicht fertiggestellt. Wenn Sie Interesse
haben, uns zu unterst�tzen, schreiben Sie uns einfach eine E-Mail:

\href{mailto:users-de@openslides.org}{users-de@openslides.org}

\section{Eigenen Webserver einsetzen}

\section{Template anpassen}
\end{document}
